In this thesis, I designed and implemented \textit{Confuzzer}, a system that
fuzzes certain classes of closed source binaries using Concolic Execution
techniques in order to find vulnerable inputs into programs that could be
leveraged by attackers to compromise systems that the binary might be running
on. The design of this system allows improved performance on fuzzing programs
that have a large branching factor or are heavily based on complex conditionals
determining control flow. The system is designed around a Taint/Crash Analysis
tool combined with a Path Exploration system to generate symbolic
representations of the paths, generating a new set of inputs to be tested. These
are implemented using a combination of Intel PIN for the Taint Analysis and
Python/z3 for the Path Exploration. Results show that while this system is very
slow in instrumenting each run of the binary, we are able to reduce the search
space to a manageable level compared to other existing tools.
