\chapter{Evaluation}
To evaluate \textit{Confuzzer}, we compare its performance on our test cases
with the anticipated performance for blind fuzzing and other systems. We can
only calculate expected performance for blind fuzzing, since it would take
longer than available to test some of our test cases.

\begin{table}
\begin{center}
\begin{tabular}{| c | c | c | c | p{6cm} |}
\hline
Test Case & Confuzzer & Blind Fuzzing & AFL & Description\\\hline
test1 & 8 & 1.10e12 & ??? & Simple magic value test\\\hline
test2 & 6 & 1.10e12 & ??? & Keygen test using xored value\\\hline
test4 & 40494 & 1.84e19 & ??? & Toy shell example without overflow\\\hline
test4 & 40494 & 1.84e19 & ??? & Toy shell example with overflow\\\hline
\end{tabular}
\end{center}
\caption{Test Cases per Fuzzing System}
\label{table:evalcases}
\end{table}

\begin{table}
\begin{center}
\begin{tabular}{| c | c | c | c | c | p{6cm} |}
\hline
Test Case & Confuzzer & Confuzzer (Distributed) & Blind Fuzzing & AFL\\\hline
test1 & 0.21 runs/sec & 3.2 runs/sec & 2513 runs/sec & 9850 runs/sec\\\hline
test2 & 0.20 runs/sec & 3.1 runs/sec & 2497 runs/sec & 9879 runs/sec\\\hline
test3 & 0.18 runs/sec & 2.7 runs/sec & 2454 runs/sec & 10.8k runs/sec\\\hline
test4 & 0.17 runs/sec & 2.7 runs/sec & 2434 runs/sec & 11.3k runs/sec\\\hline
\end{tabular}
\end{center}
\caption{Fuzzing Speed}
\label{table:evaltime}
\end{table}

We first look at the number of test cases that our fuzzers generate, since
programs of similar complexity will have roughly the same branching amounts,
giving us a better picture of how each system scales. Our rough results are in
Figure \ref{table:evalcases}. The estimate for blind fuzzing is based on the
number of possible inputs before we are likely to hit a crashing input. We
weren't able to make measurements for AFL, since it wasn't able to determine
sufficient information to prioritize certain kinds of input, even with a basic
set of example inputs for the test application. One of the advantages of AFL is
that it is able to notify the user of crashing inputs while still running,
compared to the current implementation of Confuzzer.

We then measure the amount of time each fuzzing system takes to run test cases,
to show how the speed of the Confuzzer compares to the other systems. Figure
\ref{table:evaltime} shows that the speed of the Confuzzer system is orders of
magnitudes slower than the other systems. We actually end up having better
timing from AFL than Blind Fuzzing since it forks the binary instead of creating
a new process for every test case.

This shows how the Concolic Fuzzer system is able to hone in on specific
interesting inputs without having to hit a ton of unnecessary inputs, however
the process of finding inputs takes a long time and is heavily dependent on the
size of paths in the binary being fuzzed.

While the test cases we are using for the system are a bit artificial, there are
many existing programs that have similar constructs, including parsers that
search for magic values in order to figure out whether a file is valid, key
verification systems that check whether an input is valid to detect some key
file is valid, and even network protocols that expect a certain class of inputs.

From the timing results, we could probably end up with a more efficient system
by using a combination of \textit{Confuzzer} to generate inputs that pass
through taint checks, and then using AFL or another Fuzzer in order to look for
alternate inputs past the initial checks in the program. Unfortunately doing
this sort of merged Fuzzing requires either some amount of user input to
determine when to switch between the systems, or some sort of heuristic to deal
with moving between the fuzzers.

The results for AFL also show that its possible that moving from a PIN
instrumenter to something using QEMU User Mode Emulation might offer an
improvement in the running time of the system. Most of the slowness in running
the Confuzzer system comes from the context switching and instrumentation of the
binary code. We can increase the speed of the Fuzzer using our distributed
system to run multiple instances of the binary and Taint Analysis system. Some
of the additional improvements we can apply to \textit{Confuzzer} are covered in
the \textbf{Future Work} section in Chapter 6.

% TODO: 2/4 Pages