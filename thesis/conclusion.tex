\chapter{Conclusion}
In this thesis, we presented \textit{Confuzzer}, an implementation of a Concolic
Execution Fuzzer to help with finding inputs to the program that can cause
malicious behavior. Unlike most existing Fuzzers, Concolic Execution allows us
to get through sections of the binary that require matching certain values and
allowing us to get past sections that would otherwise cause other Fuzzers to
take a long amount of time. In implementing this system, we've aimed to design
the system to allow the analysis of as many binaries as possible, without
requiring source code to perform the analysis. While this prevents us from
constructing Intermediate Representations to use in doing the taint analysis, we
are still able to parse x86 instructions sufficiently to determine how to spread
taint through the program.

While there are still many improvements that could make to the system, the
initial prototype is already able to test a class of programs that are difficult
to test with the existing systems. Being able to programatically generate test
cases is an important step in better securing existing code and doing in-depth
testing fast enough to be effective against determined attackers.
