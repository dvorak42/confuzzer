\chapter{Future Work}
While we currently have a working system, there is still a lot of work that can
be done to improve the overall system and create a more efficient and useful
system. Similar to the Design, we can split this into two major categories of
work, work for the Taint system and work for the Concolic Execution system.

\section{Taint Analysis Improvements}
One of the major improvements to the Taint Analysis system is to switch from
using Intel PIN to another instrumentation tool. Some options to explore, given
other Taint Analysis and fuzzing tools are using DynamoRIO as its support
improves, using full-system emulation, and finally using QEMU's user mode
emulation. The final option seems to be most likely to give improved results,
given the success that AFL has had with it.

In addition to switching the Taint Analysis to a different system, we can also
reduce the set of instructions that we have to examine by detecting standard
functions that we trust and instead of instrumenting the function execution, we
add special constraint equations to the Taint log. One example is being able to
detect strcmp and other related functions, and replacing them with constraints
that represent the possible return values from the functions, rather than having
tons of branching points and constraints within the library function
itself. There are a couple ways of doing this, from using some huerestic to
detect known functions, or actually stubbing out the real library we are
optimizing with one that has markers that the Taint Analysis tool would be able
to detect and choose to pause instrumentation.

Another improvement that needs to be made to the Taint Analysis system is
support for additional types of tainted input, either more general user input
from the console, to actual data packets through network sockets. Having these
additional forms of tainted input would allow us to test more parts of the code
without having to have artificial wrappers that convert a file input into a
network or user input.

Finally, the last major improvement to the Taint Analysis system that would be
useful is the addition of more ways to detect ``bad'' or potentially troublesome
states, some can be as simple as invalid memory access detection (which already
can be done with PIN) or even so far as detecting unintentional control flow
hijacking or policy violation. These would allow the system to detect a wider
array of potential issues all at once.

\section{Concolic Execution Improvements}
On the Concolic Execution end, one improvement is to the way we handle
constraints. If we are able to reduce the number of variables we create when
parsing the tainted assembly instructions, we could reduce some of the work z3
has to do when solving the equations. Similarly, if we are able to detect more
compound data structures, such as strings, we would be able to combine many of
the generated constraints together and use things like z3str for string
computations.

Additionally, if we were able to do better path and loop detection, we could
improve the search algorithm we use to search more interesting parts of the
program space without spending a bunch of time repeatedly working on the same
looping structure. Generally, having an improved search algorithm and
prioritization would allow us to reach interesting test cases much earlier.

Another overall improvement to the system that should be done in the future is
adapting it to work together with other existing Fuzzers. While
\textit{Confuzzer} has a advantage on some test cases that other fuzzers have
trouble with, the existing fuzzers also have their own advantages on other parts
of the program execution. Allowing multiple fuzzers to work together would allow
a wider search of the program input space without having all the test cases take
as long as the \textit{Confuzzer} test cases take.